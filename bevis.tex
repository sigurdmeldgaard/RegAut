\documentclass[11pt,a4paper]{article}
\usepackage{ucs}
\usepackage[utf8x]{inputenc}
\usepackage[danish]{babel}
\usepackage[T1]{fontenc}
\usepackage{graphicx}
\usepackage{amsmath}
\usepackage{amssymb}
\usepackage{hyperref}
\usepackage{url}

\title{Gennemgang af et induktionsbevis for de regulære sprogs lukkethed under reverse}
\author{Sigurd Meldgaard}
\date{24. Januar 2009}
\begin{document}
\maketitle
Bemærk at i dette dokument er alle detaljer penslet helt ud, så selvom
det ser langt ud, så skulle der kun ske een ting per skridt.
\section{Definitioner}
\subsection{Reverse funktionen}
Reverse af en streng er defineret rekursivt i strukturen af strengen:
\[rev(\Lambda) = \Lambda\]
\[rev(x\cdot a) = a\cdot rev(x), \text{ hvor } x\in \Sigma^*, a \in Sigma \]
Reverse af et sprog er defineret rekursivt som reverse af hvert enkelt element:
\[rev_L(L) = \{x | rev_s(x) \in L \} \]
\section{Hvad skal vi bevise}
Vi vil bevise at \emph{de regulære sprog er lukkede under $rev_L$}, d.v.s. hvis $L$ er regulært, så er $rev_L(x)$ også regulært.

Vi gør det ved at lave et konstruktivt bevis, d.v.s. at vi viser
hvordan man kan konstruere et regulært udtryk for $rev_L(L)$ hvis man
har et regulært udtryk for $L$.

Vi laver beviset som et induktionsbevis. D.v.s. vi viser at det gælder
for alle de måder som et regulært udtryk kan være skrevet på.
\section{Konstruktion}
Vi definerer en funktion $rev_r$ på regulære udtryk:
\subsection{Basistilfældet}
\[rev_r(\emptyset) = \emptyset\]
\[rev_r(\Lambda) = \Lambda\]
\[rev_r(a) = a, a\in\Sigma\]
\subsection{Det rekursive tilfælde}
\[rev_r(r_1 + r_2) = rev_r(r_1) + rev_r(r_2)\]
\[rev_r(r_1 \cdot r_2) = rev_r(r_2) \cdot rev_r(r_1)\]
\[rev_r(r_1^*) = rev_r(r_1)^*\]
\section{Korrekthed}
Nu har vi vist hvordan vi vil konstruere $rev_r(r)$ for alle regulære
udtryk, så skal vi bevise korrekthed, d.v.s. vise at $L(rev_r(r)) =
rev_L(L(r))$. Dette viser vi i alle 6 tilfælde.
\section{Basistilfældet}
De tre basistilfælde er nærmest trivielle:
\begin{itemize}
\item $r = \emptyset$
\[rev_r(\emptyset) = \emptyset\]

Dette er korrekt fordi
\[L(rev_r(r)) = L(rev_r(\emptyset)) = L(\emptyset) = rev_L(\emptyset) = rev_L(L(r)) \] 

Her
benytter vi at hvis vi tager $rev$ på alle elementer i den tomme
mængde har vi stadig den tomme mængde.
\item $r = \Lambda$
\[rev(\Lambda) = \Lambda\]
Dette er korrekt fordi
\[L(rev_r(r)) = L(rev_r(\Lambda)) = L(\Lambda) = \{\Lambda\} =rev_L(\{\Lambda\})
 = rev_L(L(r)) \]
 Her bruger vi definitionen på $rev(\Lambda)=\Lambda$.
\item $r = a$
\[rev(a) = a,a\in\Sigma\]
Dette er korrekt fordi (her er det skrevet helt ud i alle detaljer:
\[L(rev_r(r)) = L(rev_r(a)) = L(a) = \{a\cdot\Lambda\} \]
\[= \{a \cdot rev(\Lambda)\} = 
\{rev(\Lambda \cdot a)\} =
rev_L(\{a\})
 = rev_L(L(r)) \]
Her folder vi definitionen på $rev()$ ud en enkelt gang.
\end{itemize}
Før vi kan bevise induktionsskridtet får vi brug for to lemmaer (små
underbeviser).
\subsection{Lemma 1}
Lemmaet siger:
\[\forall x,y \in \Sigma^*: rev(xy) = rev(y)rev(x)\]
Vi beviser dette ved induktion i strukturen af $y$
\subsubsection{Basis: $y=\Lambda$}
Vi bruger definitionen af $rev$.
\[rev(xy) = rev(x\Lambda) = rev(x) = \Lambda rev(x) = rev(\Lambda)rev(x) = rev(y)rev(x)\]
\subsubsection{Induktionsskridt: $y = y'\cdot a$ hvor $y'\in \Sigma^*, a\in \Sigma$}
Vores induktionshypotese siger: $\forall x: rev(xy') = rev(y')rev(y)$

Vi bruger igen definitionen af $rev$:
\[rev(xy) = rev(xy'a) = a\cdot rev(xy')\]
Her kan vi bruge induktionshypotesen:
\[... = a\cdot rev(y')rev(x) = rev(y'a)rev(x) = rev(y)rev(x)\]
\subsection{Lemma 2}
Lemmaet siger:
\[\forall i\geq 0: rev_L(L^i) = rev_L(L)^i\]
Vi bruger induktion i $i$:
\subsubsection{Basis: $i=0$}
Alting følger her helt simpelt ved at ``pakke definitionerne ud''.
\[rev_L(L^i)=rev_L(L^0)=rev_L({\Lambda})=\{rev(\Lambda)\}=\{\Lambda\}=
rev_L(L)^0\]
I det sidste skridt skal man huske at alle sprog i nul'te er sproget med den tomme streng.
\subsubsection{Induktionsskridt: $i = i'+1$}
Induktionshypotesen: $rev_L(L^{i'}) = rev_L(L)^{i'}$. Vi kan genbruge lemma 1 her:
\[rev_L(L^i)=rev_L(L\cdot L^{i'}) = rev(L^{i'})\cdot rev(L) = rev(L)^{i'}rev(L)=
rev(L)^{i'+1} = rev(L^{i})\]
\subsection{Induktionsskridtet}
Så er vi tilbage ved hovedbeviset. Vi har tre tilfælde, og
induktionshypotesen er hver gang at: $L(rev_r(r_1)) = rev_L(L(r_1))$
og tilsvarende for $r_2$
\begin{itemize}
\item $r = r_1 + r_2$
\[rev_r(r_1 + r_2) = rev_r(r_1) + rev_r(r_2)\]
Dette tilfælde er relativt simpelt.
\[L(rev_r(r))=L(rev_r(r_1+r_2))=L(rev_r(r_1)+rev_r(r_2)) = L(rev_r(r_1))\cup L(rev_r(r2))\]
Her bruger vi induktionshypotesen:
\[\ldots = rev_L(L(r_1)) \cup rev_L(L(r_2))\]
$rev_L$ er en elementvis operation, så vi kan flytte $\cup$ indenfor parantesen:
\[\ldots = rev_L(L(r_1)\cup L(r_2)) = rev_L(L(r_1+r_2)) = rev_L(L(r))\]
\item $r = r_1\cdot r_2$
\[rev_r(r_1 \cdot r_2) = rev_r(r_2) \cdot rev_r(r_1)\]
\[L(rev_r(r)) = L(rev_r(r_2)\cdot rev_r(r_1)) = L(rev_r(r_2)) \cdot L(rev_r(r_1))\]
induktionshypotese:
\[\ldots = rev_L(L(r_2))\cdot rev_L(L(r_1)) = \{x\cdot y | xy \in rev_L(L(r_2))\cdot rev_L(L(r_1))\} \]
\[= \{rev(x)\cdot rev(y) | xy \in  L(r_2)L(r_1)\}\]
og her kan vi bruge lemma 1:
\[\ldots = \{rev(y\cdot x) | x\in L(r_2), y \in  L(r_1)\} = rev_L(L(r_2)L(r_1)) = rev_L(L(r))\]
\item $r = r_1^*$
  \[rev_r(r_1^*) = rev_r(r_1)^*\] Dette tilfælde overlades til læseren!
  Husk at bruge definitionen for $L(r^*)=\bigcup_{i=0}^{\infty}L(r)^i$
  og så lemma 2.
\end{itemize}

\end{document}